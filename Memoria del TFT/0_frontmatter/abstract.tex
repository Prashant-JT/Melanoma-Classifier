\begin{abstracts}

Las decisiones médicas son difíciles ya que menudo deben tomarse con información insuficiente e incierta. Además, el resultado del proceso de decisión tiene implicaciones de gran alcance en el bienestar humano o incluso vidas. 

El desempeño humano en la toma de decisiones disminuye con la complejidad de los problemas y la presión del tiempo. Por lo tanto, el apoyo de médicos a la toma de decisiones es crucial, especialmente en su fase inicial cuando un especialista debe elaborar un diagnóstico preliminar y especificar las posibles direcciones para el tratamiento del paciente.

Las herramientas informáticas tienen el potencial de marcar la diferencia en la medicina. Especialmente las redes profundas, que pueden aprovechar tanto el gran número de datos disponibles como la experiencia clínica.

La aplicación de redes profundas al diagnóstico se ha propuesto hace casi dos décadas, teniendo potencial para beneficiar significativamente la atención médica. Sin embargo, la difusión práctica de este enfoque sigue siendo mínima.

El objetivo de este trabajo es introducir un clasificador de diagnóstico de cáncer de piel basado en redes bayesianas utilizando Monte Carlo Dropout, propuesta por Gal \& Ghahramani en 2016 \cite{bayesian_networks-gal}. Se realiza comparaciones con varios modelos, además de mostrar otras formas de hacer frente los datos desbalanceados en datos clínicos frente a las técnicas clásicas. También se muestra la conveniencia de calibrar los modelos de clasificación para que las probabilidades mostradas en el diagnóstico se ajusten a los datos.

Finalmente, se argumenta que un diagnóstico con la ayuda informática puede ser beneficiosa para sus usuarios y mejorar la precisión diagnóstica de los médicos.

\end{abstracts}
