\pagestyle{fancy}
\fancyhf{}
\fancyhead[L]{1.1 MOTIVACIÓN}

\subsection{Motivación}
El cáncer de piel está aumentando en muchos países, debido a que cada vez la capa de ozono que protege a la tierra de la radiación solar se debilita. Uno de los principales responsables del cáncer de piel de melanoma es la radiación ultravioleta proveniente del sol, o de cualquier otra fuente artificial de radiación ultravioleta, como las lámparas bronceadoras.

El Índice de Radiación Ultravioleta (Índice UV o UVI) se mide en una escala a partir de 0 sin límite superior, en la que un índice UV menor de 2, representa un riesgo bajo (código de color verde) y un índice UV mayor de 11 representa un riesgo extremadamente alto (código de color violeta).

En Canarias, no solo el sol acompaña todo el año con una exposición constante, también una exposición estacional; es decir, personas que durante todo el año no toman demasiado sol pero que en verano sí lo hacen de forma constante. Al sol de Canarias también se suman los niveles de radiación que existe en la región dada su proximidad al Ecuador. Esto hace que el índice de radiación UV sea extremadamente elevado, siendo el más alto de las Comunidades Autónomas de España.
\newpage

FIGURA ESPAÑA

\bigskip
Por lo que, la motivación principal del trabajo ha sido la identificación e incorporación de los factores que influyen en el cáncer de piel para obtener el valor predictivo positivo a partir de dichos factores y del resultado de un clasificador basado en redes neuronales convolucionales.
\newpage

\pagestyle{fancy}
\fancyhf{}
\fancyhead[L]{1.2 OBJETIVOS y 1.3 METOLOGÍA}
\subsection{Objetivos}
El objetivo principal es apoyar el diagnóstico médico, ayudar a los médicos para tomar mejores decisiones y detectar la enfermedad en una fase tempranera con el propósito de poder realizar el tratamiento adecuado. Para ello se implementa un clasificador que sea capaz de discernir imágenes de cáncer maligno (particularmente, de tipo melanoma) y benigno. Asimismo, implementar una red bayesiana que permita identificar e incorporar otros factores que influyen en el cáncer de piel.

\subsection{Metodología}
El desarrollo se ha hecho principalmente en la plataforma de Kaggle utilizando Jupyter Notebook, en el lenguaje de programación Python. Respecto a la modelación de la red bayesiana se ha hecho uso de la plataforma ------------.