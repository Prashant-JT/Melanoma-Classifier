\chapter{Diagnosis en \\ Hematología/Oncología}
La Hematología/Oncología es la rama de la medicina que se centra en el diagnóstico y tratamiento de numerosos trastornos sanguíneos, incluidos el cáncer, la anemia, la leucemia y la enfermedad de Hodgkin, y el diagnóstico y tratamiento de cánceres y tumores benignos y malignos.

La mejor manera de detectar el melanoma es examinando continuamente la piel de los pacientes, especialmente los lunares. Una llaga, una protuberancia o un tumor en la piel también puede ser un signo de melanoma u otro cáncer de piel. El melanoma se puede encontrar en varios lugares incluyendo la espalda, las nalgas, las piernas, el cuero cabelludo, el cuello, detrás de la oreja, las plantas de los pies, las palmas de las manos, dentro de la boca, los genitales y debajo de las uñas \cite{utmedicalcenter}. 

Según la Academia Americana de Dermatología (AAD), aproximadamente del 20\% al 40\% de los melanomas se desarrollan a partir de un lunar. Una llaga o tumor que sangra o cambios en el color de la piel también puede ser un signo de cáncer de piel \cite{utmedicalcenter}. 

%\newpage

La clave para tratar con éxito el melanoma es reconocer los síntomas a tiempo. Se recomienda realizar exámenes corporales anuales por un dermatólogo y examinarse la piel una vez al mes. Además, si un paciente ha tenido cáncer de piel, debe hacerse chequeos regulares para que un médico pueda examinar su piel.
Hay un número de pruebas que se pueden ordenar para diagnosticar el cáncer de piel: 

\begin{itemize}
\item Biopsia
\item Tomografía computarizada (TC)
\item Imagen de resonancia magnética (IRM)
\item Tomografía por emisión de positrones (TEP)
\end{itemize}

Por lo que un clasificador que sea capaz de discernir entre imágenes de piel de cáncer maligno y benigno ayudaría al diagnóstico mensual que deben hacer los médicos a sus pacientes además de reconocer los síntomas a tiempo.